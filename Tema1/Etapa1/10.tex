\documentclass[runningheads]{llncs}
\usepackage[T1]{fontenc}
\usepackage{graphicx}

\begin{document}

\title{Compresia Datelor}
\author{Stoica Robert-Valentin}
\authorrunning{Stoica Robert- Valentin}

\institute{Universitatea Politehnica, Bucuresti}

\maketitle

\begin{abstract}
Scopul acestei lucrari este acela de a aduce in atentie si a analiza
eficienta algoritmilor de compresie Huffman si Lempel-Ziv-Welch (LSW)
pe un set cat mai amplu de date.

\keywords{Compresie  \and Huffman Coding \and Lempel-Ziv-Welch}
\end{abstract}



\section{Intorducere}

\subsection{Descrierea problemei}
Compresia datelor este un aspect foarte important atunci cand vorbim de
lucrul cu fisiere de orice tip (audio, video, text etc) deoarece permite
partajarea si pastrarea lor intr-o forma redusa, simplificata, putand astfel
economisi resurse.

Folosirea fisierelor comprimate in locul celor originale ne ofera atat eficienta
in cadrul utilizarii memoriei fizice (pastrarea lor intr-o forma redusa va ocupa
mai putin spatiu pe HDD/ SSD) cat si una temporala in cazul in care dorim sa
partajam resurse prin intermediul internetului, un fisier de dimensiune mai mica 
fiind incarcat si transferat mult mai usor.

\subsection{Specificarea soluțiilor alese}

Cele doua metode de compresie ce urmeaza a fi analizate fac parte din clasa
"lossless data compression" (compresie fara pierdere a datelor), asta insemnanad
ca, in urma procesului de decomprimare, fisierul poate fi adus la forma sa 
initiala fara a suferi modificari de orice tip.

In cazul compresiei Huffman, algoritmul creeaza un cod binar unic pentru fiecare
caracter (sau octet) existent in cadrul fisierului. Acest cod va fi mai scurt
sau mai lung in functie de frecventa fiecarui caracter.
Astfel, caracterele ce apar foarte des vor avea un cod binar \textbf{scurt} iar cele ce
apar rar, vor avea unul mult mai \textbf{lung}. "Magia" acestui algoritm sta in capacitatea
sa de a compensa codurile scurte ce apar de multe ori cu codurile lungi ce apar mult
mai rar, in medie acesta reusind sa aduca fisierul la o dimensiune mai mica decat cea
originala.

In schimb, metoda de compresie Lempel-Ziv-Welch aduce in plus posibilitatea de
a codifica \textbf{secvente de mai multe caractere ce se repeta}. Algoritmul aduce cu sine
un dictionar ce isi mareste capacitatea cu fiecare secventa noua gasita.
Desi codificarile vor fi facute pe un numar fix de biti (in general mai mare decat
dimensiunea unui octet, ajungand pana la 12 sau 13 biti), daca fisierul are cuvinte
sau secvente de litere ce se repeta, acestea vor aparea in fisierul comprimat sub forma
unui singur cod.


\subsection{Evaluarea soluțiilor}
Singurul aspect ce ne intereseaza in cazul algoritmilor de compresie este dat de raportul
dintre dimensiunea fisierului comprimat si cea a fisierului original, altfel spus, eficienta
algoritmilor consta in \textbf{procentul de reducere} al fisierului original.

Pentru evaluarea eficientei algoritmilor voi incerca sa acopar o plaja cat mai mare a
datelor de intrare atat prin tipul, dimensiunea dar si continutul lor.

In cazul fisierelor text, voi genera date de intrare mai mult sau mai putin favorabile
pentru ambele tipuri de compresie. Un exemplu bun in cazul fisierelor text ar fi un fisier
ce contine \textbf{o singura litera de un numar mare de ori}:

\begin{itemize}
    \item in cazul algoritmului Huffman, aceasta compresie este \underline{foarte buna},
    in noul fisier retinandu-se un singur cod foarte scurt (+ datele auxiliare necesare decomprimarii)
    \item pentru algoritmul LZW, compresia este una \underline{mult mai slaba}, acesta adaugand grupuri
    de aceeasi litera dar dimensiuni diferite in dictionar si in fisierul nou rezultat

    Totusi, in ambele cazuri, fisierul comprimat va fi \textbf{mai mic} decat cel initial.
\end{itemize}

Un alt exemplu ar putea fi un text ce contine \textbf{caractere complet diferite}. In acest caz,
pentru date de intrare mai mari, ambii algoritmi vor fi \textbf{extrem de ineficienti}, existand
posibilitatea ca noul fisier sa fie chiar mai mare decat cel original.

Testele vor fi facute atat pentru cazuri bine alese (ca cele de mai sus), cat si pentru
fisiere generate aleator (text copiat de undeva sau imagini care nu au un tipar anume).

O alta abordare ar putea fi data de testarea algoritmilor pe date de intrare care deja sunt
supuse unui grad mare de compresie (imagini, fisiere PDF etc), moment in care ne asteptam ca
algoritmii sa aiba un impact \textbf{nesemnificativ} (sau chiar unul rau - dimensiunea fisierului creste)
asupra fisierului de intrare.

Pentru o parte din teste voi genera date \textbf{complet aleatoare} folosind comanda:
cat /dev/urandom.

\subsubsection{Scopul lucrarii} este acela de a analiza toate aceste situatii si a observa
in ce caz este mai bun un algoritm decat celalalt.



%%% TODO
\begin{thebibliography}{8}
\bibitem{ref_article1}
Autor, Gajendra Sharma, Analysis of Huffman Coding and Lempel–Ziv–Welch (LZW) Coding as
Data Compression Techniques\textbf{}

\bibitem{ref_url1}
\url{https://www.geeksforgeeks.org/huffman-coding-greedy-algo-3/}

\bibitem{ref_url2}
\url{https://www.programiz.com/dsa/huffman-coding}

\bibitem{ref_url3}
\url{http://www.cas.mcmaster.ca/~cs2c03/2020/LN21-2020.pdf}

\bibitem{ref_url4}
\url{https://www.youtube.com/watch?v=j2HSd3HCpDs\&t=428s}

\bibitem{ref_url5}
\url{https://www.youtube.com/watch?v=dM6us854Jk0}

\bibitem{ref_url6}
\url{https://www.youtube.com/watch?v=JsTptu56GM8}

\end{thebibliography}
\end{document}
